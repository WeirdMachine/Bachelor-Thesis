\chapter{Load Balancing with and for Kubernets}

Now that we know some basics let's take a deeper look inside the Service object and external Load Balancing.

\section{kube-proxy}\label{sec:kubeproxy}

As mentioned in the previous chapter, kube-proxy reflects services on each node.
It implements a network proxy and a load balancer, as well as a virtual ip for services, that is used for routing inside the cluster.
In general kube-proxy watches the control plane for changes to services and endpoints objects and opens ports on the host machine.
The behaviour of the load balancer implementation as well as the proxy can be controller via the proxy-mode option.
Kube-proxy also takes on the task of forwarding traffic that hits a node where the application is not deployed to the appropriate node.~\cite{KUBERNETES-SERVICE}

\subsection{User space proxy mode}

In this mode kube-proxy takes care of routing and acts as a central part.
\\
Incoming connections are routed through kube-proxy and forwarded to one of the corresponding endpoints.
The endpoint is chosen by default via a round-robin algorithm.
\\
For internal routing it installs iptable rules for the clusterIp and port.
These rules redirect the traffic to the proxy port, which proxies the Pod.
\\
In terms of failures kube-proxy can detect this and retry routing to another pod.
\\
This mode has some downsides because it needs to switch between userspace and kernel space.
Due to the performance loss and problems with reliability this is not a recommended mode.~\cite{KUBERNETES-SERVICE}

\subsection{iptables proxy mode}

This mode completely rely on iptables and do not proxy all traffic directly through kube-proxy.
For each service kube-proxy installs a set of iptable rules, which capture traffic to the Service's clusterIp and port, and forward the traffic to the Pod.
The endpoint by default is chosen randomly.
\\
Iptables won't recognize failures like timeouts.
Kubernetes offers readiness probes for Pod's that tell kube-proxy which Pod's are healthy and hide unhealthy Pods.
\\
Because of the use of the low-level iptables, this method has less overhead and does not completely rely on kube-proxy.~\cite{KUBERNETES-SERVICE}

\subsection{IPVS proxy mode}

IPVS (IP Virtual Server) is a linux module built on top of iptables.
It was developed for load balancing within the Linux kernel and offers a significant performance increase for large clusters compared to iptables.
With a certain number of rules, iptables loses a lot of performance, because all rules have to be processed.
There are scenarios that IPVS cannot handle, for example NodePort type service.
Then iptables are used as fallback, but with ipset\footnote{https://ipset.netfilter.org/} to keep the number of rules low.
\\
Since IPVS comes with a load balancer implementation, it also offers several load balancing algorithms to choose from.
Among them are ``Round Robin``, ``least connection``, ``shortest expected delay`` and more.~\cite{KUBERNETES-SERVICE}

\section{External Load Balancer}\label{sec:ExternalLoadBalancer}

Kubernetes services offer two ways to expose an application to the outside world.
These are set via the ``.spec.type`` field inside the service object.
\\
If the type is not specified, ClusterIp is set as default and the service is only accessible within the cluster.
\\
With the type NodePort, a random port, by default in between of 30000-32767, is chosen and opened on all nodes of the cluster.
Thus, the service is externally accessible on all nodes of the cluster via the allocated port.
At this point you could manually configure an external load balancer which uses the nodes and the appropriate port as endpoints.
But since the node ports can change, because they are usually allocated randomly, it makes sense to automate this process, also to avoid manual configuration which may cause human errors.
So let's introduce the type LoadBalancer, which makes the service in principle just like type NodePort externally accessible, with the small difference that an external load balancer is requested and a status field is set.
Since the LoadBalancer is not part of Kubernetes, it requires an external configuration of the load balancer.
Within the Kubernetes cluster, a controller is installed which watches for the type LoadBalancer and then, depending on the infrastructure setup, creates or configures an external load balancer and sets the status within the cluster.
The status field should be updated with an allocated IP address of the load balancer, that is used to access the application from the outside.~\cite{KUBERNETES-SERVICE}

\section{Ingress Controller}\label{sec:IngressController}

An Ingress Controller is an application that usually runs inside a Kubernetes cluster and offers features of Layer 7 load balancing.
It watches for ingress objects and configures itself accordingly.
They usually serve as entrypoints for HTTP applications within the cluster.
Therefore, it is common to have a service of type LoadBalancer that points to the pods of the Ingress Controller.
This then forwards the traffic to the appropriate pods based on rules from an Ingress object.
Thus, applications do not have to be directly exposed via a service, but can be appropriately routed via an Ingress object.
In addition to the features of the Ingress Controller, this method also offers a more dynamic approach, since the entry point into the cluster only has to be defined once.
TLS termination can also be performed at the ingress controller, which simplifies certificate management.~\cite{KUBERNETES-INGRESS}
\\
An ingress controller, like the external load balancer, is not installed by default.
There are several implementations of the Ingress Controller that can be installed.~\cite{KUBERNETES-INGRESS-CONTROLLER}

\section{Bare metal vs. cloud}

As mentioned in the previous sections, things like external load balancers and ingress controller require some manual setup.
\\
In a cloud environment with managed Kubernetes, the cloud provider typically installs a custom cloud controller manager inside the clusters control plane.
The cloud controller manager takes care of the connection of cloud-specific api's via which, for example, an external load balancer is configured.
Also ingress controllers can be installed automatically without any manual effort.
\\
In a bare metal environment, there is no one-size-fits-all solution, as it is highly dependent on the surrounding infrastructure.
There are a number of open source solutions such as MetallB or kube-vip, but these require manual setup.
