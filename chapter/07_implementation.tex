\chapter{Implementation}

This chapter is about implementation specifics.
KubeLB is written in the Go\footnote{https://golang.org/} programming language.
%Todo: More detailes about why go and why the libraries we use are also in go
Since Kubernetes and its ecosystem are also written in Go, integration is easiest here.
A basic understanding of the programming language is assumed.

\section{KubeBuilder}

KubeBuilder is a framework to build Kubernetes APIs using Custom resource definitions, build on top of the controller-runtime and controller-tools libraries.
It is a good entrypoint to start developing an Operator, as it simplifies CRD creation and controller implementation.
Like other frameworks it provides the developer a simple abstraction and reduces boilerplate and toil.
\\
KubeBuilder can be used to initialize a new project, initiating a basic Go project structure, as well as several configuration files to deploy within a Kubernetes cluster.
To build a controller, various modules are required, which are added by KubeBuilder to the Project.
\\
For controller specific components the controller-runtime\footnote{https://github.com/kubernetes-sigs/controller-runtime} module provides an abstraction layer.
The interaction with Kubernetes is done via the Kubernetes API which is implemented within the client-go\footnote{https://github.com/kubernetes/client-go} module.
The controller-runtime module provides an abstraction for various components to build an Operator.

\begin{itemize}
    \item \textit{Manager} \\
    The Manager configures the go-client, a cache and is generally responsible for the management of shared resources.
    Several controllers can be registered at a Manager, these can be started or stopped over the manager.
    In addition, the manager takes over the leader election within a cluster and thus ensures regulated behavior and fail-safety.
    \item \textit{Controller} \\
    Controllers use events to trigger reconcile requests.
    They can trigger reconcile requests based on Predicates that filter events.
    \item \textit{Reconciler} \\
    Controller logic is implemented in terms of Reconcilers.
    A Reconciler implements a function which takes a reconcile request containing the name and namespace of the object to reconcile.
    It returns a Result or an error, indicating weather to requeue the Request.
\end{itemize}


\autoref{lst:main-manager} illustrates how the three components interact together in the actual application.
\\
At the beginning of the application a Manager and the Reconciles are created.
The Manager creates a client by the given configuration, that \textit{ctrl.GetConfigOrDie()} provides.
%Todo: explain scheme or remove it
When creating the KubeLbNodeReconciler, the client and scheme are used from the Manager.
\\
\textit{SetupWithManager()} creates a new Controller and attach the Reconciler, the Controller is than registered at the Manager.
Also the object to watch for is declared in the function as shown in \autoref{lst:nodecontroller-reconcile}, that watches for Nodes.
\\
In the end, \textit{Start()} is called at the manager, which starts all registered Controllers.

\begin{lstlisting}[caption={KubeLB Agent main.go snippet - Manager and Controller}, label={lst:main-manager}]

	mgr, err := ctrl.NewManager(ctrl.GetConfigOrDie(), ctrl.Options{
		Scheme:             scheme,
		MetricsBindAddress: metricsAddr,
		Port:               9443,
		LeaderElection:     enableLeaderElection,
		LeaderElectionID:   "k8c.io.kubelb.agent",
	})
	if err != nil {
		setupLog.Error(err, "unable to start agent")
		os.Exit(1)
	}

	if err = (&agent.KubeLbNodeReconciler{
		Client:    mgr.GetClient(),
		Log:       ctrl.Log.WithName("kubelb.node.reconciler"),
		Scheme:    mgr.GetScheme(),
		KlbClient: kubeLbClient.TcpLbClient,
		Endpoints: &sharedEndpoints,
	}).SetupWithManager(mgr); err != nil {
		setupLog.Error(err, "unable to create controller", "reconciler", "kubelb.node.reconciler")
		os.Exit(1)
	}

	// +kubebuilder:scaffold:builder
	if err := mgr.Start(ctx); err != nil {
		setupLog.Error(err, "problem running agent")
		os.Exit(1)
	}


\end{lstlisting}

The Controller eventually triggers a reconcile on node changes.
\\
To save load on the Kubernetes API, the Agent keeps an internal list of the current endpoints (\textit{kubelb.Endpoints}) of the cluster.
This means that the API does not have to be queried for the current nodes every time.
One task of the KubeLbNodeReconciler is to update this list when changes occur.
\\
In order for existing load balancers to be aware of node changes in the user cluster, it is necessary to change the endpoints in the Spec of all TCPLoadBalancer objects.
To do this, all TCPLoadBalancers are queried via the KubeLB client, which has a configuration for KubeLB cluster, and their endpoints are set to the updated ones.
\\
The changes are registered by the Manager in the KubeLB cluster, which then passes the configuration to the Envoy load balancers as explained in more detail in \autoref{sec:envoy-control-plane}.
\\
\begin{lstlisting}[caption={KubeLB Agent node reconciler}, label={lst:nodecontroller-reconcile}]
import (
	"context"
	"github.com/go-logr/logr"
	kubelbiov1alpha1 "k8c.io/kubelb/pkg/api/kubelb.k8c.io/v1alpha1"
	"k8c.io/kubelb/pkg/generated/clientset/versioned/typed/kubelb.k8c.io/v1alpha1"
	"k8c.io/kubelb/pkg/kubelb"
	corev1 "k8s.io/api/core/v1"
	v1 "k8s.io/apimachinery/pkg/apis/meta/v1"
	"k8s.io/apimachinery/pkg/runtime"
	ctrl "sigs.k8s.io/controller-runtime"
	"sigs.k8s.io/controller-runtime/pkg/client"
)

// KubeLbIngressReconciler reconciles a Service object
type KubeLbNodeReconciler struct {
	client.Client
	KlbClient v1alpha1.TCPLoadBalancerInterface
	Log       logr.Logger
	Scheme    *runtime.Scheme
	Endpoints *kubelb.Endpoints
}

// +kubebuilder:rbac:groups="",resources=nodes,verbs=list;get;watch

func (r *KubeLbNodeReconciler) Reconcile(ctx context.Context, req ctrl.Request) (ctrl.Result, error) {
	log := r.Log.WithValues("name", req.Name)

	log.V(2).Info("reconciling node")

	nodeList := &corev1.NodeList{}
	err := r.List(ctx, nodeList)

	if err != nil {
		log.Error(err, "unable to list nodeList")
		return ctrl.Result{}, err
	}

	log.V(6).Info("processing", "nodes", nodeList, "endpoints", r.Endpoints)

	if r.Endpoints.EndpointIsDesiredState(nodeList) {
		log.V(2).Info("endpoints are in desired state")
		return ctrl.Result{}, err
	}

	log.V(6).Info("actual", "endpoints", r.Endpoints.ClusterEndpoints)
	log.V(6).Info("desired", "endpoints", r.Endpoints.GetEndpoints(nodeList))

	r.Endpoints.ClusterEndpoints = r.Endpoints.GetEndpoints(nodeList)
	log.V(5).Info("proceeding with", "endpoints", r.Endpoints.ClusterEndpoints)

	//patch endpoints
	tcpLbList, err := r.KlbClient.List(ctx, v1.ListOptions{})
	if err != nil {
		log.Error(err, "unable to list TcpLoadBalancer")
		return ctrl.Result{}, err
	}

	log.V(6).Info("patching", "TcpLoadBalancers", tcpLbList)

	var endpointAddresses []kubelbiov1alpha1.EndpointAddress
	for _, endpoint := range r.Endpoints.ClusterEndpoints {
		endpointAddresses = append(endpointAddresses, kubelbiov1alpha1.EndpointAddress{
			IP: endpoint,
		})
	}

	for _, tcpLb := range tcpLbList.Items {
		for _, endpoints := range tcpLb.Spec.Endpoints {
			endpoints.Addresses = endpointAddresses
		}

		_, err = r.KlbClient.Update(ctx, &tcpLb, v1.UpdateOptions{})
		if err != nil {
			log.Error(err, "unable to update", "TcpLoadBalancer", tcpLb.Name)
		}
		log.V(2).Info("updated", "TcpLoadBalancer", tcpLb.Name)
		log.V(7).Info("updated to", "TcpLoadBalancer", tcpLb)

	}

	return ctrl.Result{}, nil
}

func (r *KubeLbNodeReconciler) SetupWithManager(mgr ctrl.Manager) error {
	return ctrl.NewControllerManagedBy(mgr).
		For(&corev1.Node{}).
		Complete(r)
}

\end{lstlisting}

\section{Code generation}\label{sec:code-generator}

Due to the lack of generics in go, it is common to use code generators.
The controller-tools\footnote{https://github.com/kubernetes-sigs/controller-tools} repository includes the controller-gen command which is used for generating utility code and Kubernetes YAML.
In addition, the code-generator\footnote{https://github.com/kubernetes/code-generator} repository, is used for Kubernetes client generation.
Both repositories contain a CLI which can perform different types of generation, that are mostly based on special marker comments in the Go code.

\subsection{RBAC}

As described in \autoref{sec:agent}, the Agent needs access to some Kubernetes resources like services.
The permissions are closely coupled to the controller implementation.
For reasons of clarity, it therefore makes sense to store the required permissions close to the code base.
In \autoref{lst:nodecontroller-reconcile}, the markers are set above the reconcile method, that implements the controller logic.
The controller needs read-only access because it reacts to changes in the nodes and needs to get them from the api server.
\\
With the RBAC markers set, controller-gen can generate the agent role, that is used to run the Agent inside a Kubernetes cluster.
The command below creates the agent-role inside the output path, based on the marker comments controller-gen finds in the set path.
\\
\begin{lstlisting}[numbers=none, caption={Generate Role YAML files with controller-gen}, label={lst:role-generation}]
controller-gen rbac:roleName=agent-role paths="./pkg/controllers/agent/..." output:artifacts:config=config/agent/rbac
\end{lstlisting}

\subsection{Custom Resource Definitions}

KubeBuilders "create api" command creates the basic structure for the CRDs.

The \autoref{lst:tcplb} shows the basic struct in go of the TCPLoadBalancer CRD.
Like all Objects in Kubernetes it contains Type-, and ObejectMeta Information, a Spec and in this case also a Status field.
The struct is annotated with marker comments for the controller-gen tool, which is than able to create a YAML file to deploy the CRD inside a Cluster.
\\
The first three comments are for CRD generation, while \textit{+genclient} is needed for client generation, which is explained in \autoref{subsec:client}.
The annotations express that this is the root object, it includes a status field and the abbreviation \textit{tcplb}.
For serialization json tags are required.
%Todo: footnote json seriali

\begin{lstlisting}[caption={TCPLoadBalancer CRD root struct},label={lst:tcplb}]
import metav1 "k8s.io/apimachinery/pkg/apis/meta/v1"

// +kubebuilder:object:root=true
// +kubebuilder:subresource:status
// +kubebuilder:resource:shortName=tcplb
// +genclient

// TCPLoadBalancer is the Schema for the tcploadbalancers API
type TCPLoadBalancer struct {
	metav1.TypeMeta   `json:",inline"`
	metav1.ObjectMeta `json:"metadata,omitempty"`

	Spec   TCPLoadBalancerSpec   `json:"spec,omitempty"`
	Status TCPLoadBalancerStatus `json:"status,omitempty"`
}
\end{lstlisting}

\autoref{lst:tcpbl-status} is the implementation of the TCPLoadBalancer status field.
KubeLB mirrors the Kubernetes Service status and for that reason it makes use of the preexisting LoadBalancerStatus from the api module.
The LoadBalancer field is marked as optional, because it is absent if there is no load balancer provisioned yet.

\begin{lstlisting}[caption={TCPLoadBalancerStatus struct}, label={lst:tcpbl-status}]
import corev1 "k8s.io/api/core/v1"

// TCPLoadBalancerStatus defines the observed state of TCPLoadBalancer
type TCPLoadBalancerStatus struct {
	// LoadBalancer contains the current status of the load-balancer,
	// if one is present.
	// +optional
	LoadBalancer corev1.LoadBalancerStatus `json:"loadBalancer,omitempty"
}
\end{lstlisting}

Every load balancer needs a set of Endpoints, as well as Ports to expose to the outside of the cluster.
It is also possible to set the Type, like in a Kubernetes service.
In any case at least one Endpoint is required, the "//+kubebuilder:validation:MinItems:=1" annotation ensures that the generated Resource contains a validator.
\\
The command in \autoref{lst:crd-generation} will check the ./pkg directory for marker comments and create the CRD of version v1 inside the config/crd/base directory.
\\
\begin{lstlisting}[caption={TCPLoadBalancerSpec struct}, label={lst:tcpbl-spec}]
// TCPLoadBalancerSpec defines the desired state of TCPLoadBalancer
type TCPLoadBalancerSpec struct {
	// Important: Run "make" to regenerate code after modifying this file

	// Sets of addresses and ports that comprise an exposed user service on a cluster.
	// +required
	//+kubebuilder:validation:MinItems:=1
	Endpoints []LoadBalancerEndpoints `json:"endpoints,omitempty"`

	// The list of ports that are exposed by the load balancer service.
	// +optional
	Ports []LoadBalancerPort `json:"ports,omitempty"`

	// type determines how the Load Balancer Service is exposed. Defaults to ClusterIP. Valid
	// options are ClusterIP, NodePort, and LoadBalancer.
	// +optional
	// +kubebuilder:default:=ClusterIP
	Type corev1.ServiceType `json:"type,omitempty"
}
\end{lstlisting}

\begin{lstlisting}[numbers=none, caption={Generate CRD YAML files with controller-gen}, label={lst:crd-generation}]
	controller-gen crd:crdVersions=v1 paths="./pkg/..." output:crd:artifacts:config=config/crd/bases
\end{lstlisting}

\subsection{Client}\label{subsec:client}

The go-client module includes an implementation for the standard Kubernetes objects.
Since KubeLB extends the Kubernetes API with a CRD, an implementation in Go is also required for the Agent (\autoref{sec:agent}), to programmatically interact with the new resource.
In order for the Agent controller to create a watch on the CRD, it needs an informer.
The informer stores objects which it receives from the client and invoke the controller passing it the object.
Therefore, the informer still needs a client as well as a lister to work.
The client communicates with the Kubernetes API and creates watches.
The lister is an abstraction to get and list the respective CRD, which is needed by the informer.
\\
The \textit{+genclient} marker comment indicates to create a client for the Custom Resource, like in \autoref{lst:tcplb}.
Within the code-generator project there are several binaries, like client-gen, lister-gen and informer-gen, which generate different parts of code.
To bundle them the project offers a bash script called generate-groups.sh, wich acts as an entrypoint and calls the binaries.
It offers the ability to generate a client, lister and informer.
\autoref{lst:generate-groups} illustrates the usage of the command.
The first arguments are the generators to be invoked by the script.
The second and third ones are output and input modules, and the last one is the group version of the CRD.

\begin{lstlisting}[numbers=none, caption={Generate client, informer and lister with code-generator}, label={lst:generate-groups}]
	generate-groups.sh "client,lister,informer" k8c.io/kubelb/pkg/generated k8c.io/kubelb/pkg/api "kubelb.k8c.io:v1alpha1"
\end{lstlisting}

\section{Envoy control-plane}\label{sec:envoy-control-plane}

As explained in \autoref{sec:envoy}, Envoy was chosen as a load balancer among other things because of the Data plane API.
The Envoy data plane api\footnote{https://github.com/envoyproxy/data-plane-api} is implemented by the go-control-plane\footnote{https://github.com/envoyproxy/go-control-plane} module.
\\
The module provides a snapshot cache which needs to be initialized, updated and invalidated, as well as a grpc based api server implementation.
To expose the Envoy data-plane api server to the cluster, the manager deployment consist of a service.
\\
An envoy load balancer deployment is created for every TCPLoadBalancer.
Each envoy deployment gets a bootstrap configuration which tells envoy where to find the control-plane, as well as a unique node id.
This way envoy will connect to the control-plane and serve the current configuration for its node id.
\\
The Module uses snapshots to represent envoy configurations, that are stored in a cache.
A controller inside the Manager is watching for changes to the TCPLoadBalancer resources and reconciles multiple Kubernetes Objects like service and deployments.
Although the envoy snapshot cache is not living inside Kubernetes, the Manager follows the same approach and reconciles the envoy snapshot like the other resources.
The function is called by the TCPLoadBalancerReconciler and gets the current TCPLoadBalancer passed.
Snapshots are identified by a unique name and a version.
The actual snapshot is the latest version inside the snapshot cache.
The snapshot created from the provided and probably updated TCPLoadBalancer object is the desired one.
If no snapshot is present, it gets simply initialized with the desired one.
Otherwise, if a change is detected it will update the snapshot and increase the version.
The comparison is done based on the desired and actual snapshots, if they differ the snapshot cache needs to be updated.
\\

\begin{lstlisting}[caption={Envoy snapshot reconciliation}, label={lst:snapshot-reconcile}]
func (r *TCPLoadBalancerReconciler) reconcileEnvoySnapshot(ctx context.Context, tcpLoadBalancer *kubelbk8ciov1alpha1.TCPLoadBalancer) error {

	log := ctrl.LoggerFrom(ctx).WithValues("reconcile", "envoy")
	log.V(2).Info("verify envoy snapshot")

	// Get current snapshot
	actualSnapshot, err := r.EnvoyCache.GetSnapshot(tcpLoadBalancer.Name)
	if err != nil {
		// Add new snapshot to the cache
		initSnapshot := envoycp.MapSnapshot(tcpLoadBalancer, "0.0.1")
		log.Info("init snapshot", "service-node", tcpLoadBalancer.Name, "version", "0.0.1")
		log.V(5).Info("serving", "snapshot", initSnapshot)

		return r.EnvoyCache.SetSnapshot(tcpLoadBalancer.Name, initSnapshot)
	}

	log.V(5).Info("actual", "snapshot", actualSnapshot)

	// Generate a new snapshot using the old version to be able to do a DeepEqual comparison
	lastUsedVersion, err := semver.NewVersion(actualSnapshot.GetVersion(envoyresource.ClusterType))
	if err != nil {
		return errors.Wrap(err, "failed to parse version from last snapshot")
	}

	desiredSnapshot := envoycp.MapSnapshot(tcpLoadBalancer, lastUsedVersion.String())
	log.V(5).Info("desired", "snapshot", desiredSnapshot)

	if reflect.DeepEqual(actualSnapshot, desiredSnapshot) {
		log.V(2).Info("snapshot is in desired state")
		return nil
	}

	newVersion := lastUsedVersion.IncMajor()
	newSnapshot := envoycp.MapSnapshot(tcpLoadBalancer, newVersion.String())

	if err := newSnapshot.Consistent(); err != nil {
		return errors.Wrap(err, "new Envoy config snapshot is not consistent")
	}
	log.Info("updating snapshot", "service-node", tcpLoadBalancer.Name, "version", newVersion.String())

	if err := r.EnvoyCache.SetSnapshot(tcpLoadBalancer.Name, newSnapshot); err != nil {
		return errors.Wrap(err, "failed to set a new Envoy cache snapshot")
	}

	return nil
}
\end{lstlisting}
