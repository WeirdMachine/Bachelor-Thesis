\chapter{One step further}

In this chapter, an outlook on further features for the KubeLB Project will be given.

\section{TLS}

TLS termination is a common feature for L7 load balancers and supported by Envoy.
For this it is necessary to provide key material to Envoy.
Kubernetes provides the secret of type \textit{kubernetes.io/tls} for this purpose, which contains a TLS key pair.
This secret can be mounted in multiple pods and used by Envoy for TLS termination.
\\
To create the secret and the key material, there is the possibility of automated solutions like Let's encrypt\footnote{https://letsencrypt.org/}.
In conjunction with Cert-manager\footnote{https://cert-manager.io/}, this enables simple integration of automatically generated certificates, which can be ensured by KubeLB Manager.
Secret management systems such as Vault\footnote{https://www.vaultproject.io/} can also be integrated into KubeLB Manager.
\\
In case the customer has a certificate without using an internal secret management, a manual creation of the secret within the KubeLB cluster could also be a solution.
An integration of user cluster secrets made usable in LB-CLuster is also an option, even if it is not the best solution from a security point of view.

\section{Multi-cluster Service Load Balancing}

Kubernetes does not offer any concept of inter cluster communication.
However, there are many good reasons to split an application across multiple clusters.
In a scenario where one application is deployed to multiple clusters, a load balancer is required to dedicate the traffic across the clusters.
KubeLB could be a solution and provide the load balancer instance with the proper endpoints set.
\\
Today's approaches use a service mesh like istio\footnote{https://istio.io/latest/docs/concepts/what-is-istio/}, which gives the cluster operator complete control of the traffic flow.
However, these are not trivial to configure and might have a big impact on the cluster, since they affect the entire networking.
\\
The KEP-1645\footnote{https://github.com/kubernetes/enhancements/tree/master/keps/sig-multicluster/1645-multi-cluster-services-api} (Kubernetes Enhancement Proposal) of sig-multicluster, suggest an implementation of a multi cluster service api to the Kubernetes Core.
This should help with the implementation for KubeLB and could be a convenient and clean way for a multi-cluster service load balancing solution, without any third party requirements.
