\chapter{Testing}

This chapter covers unit tests, and the advanced controller tests of the KubeLB project.

\section{Unit test}\label{sec:unit-test}
In general unit tests are used to test independent parts or units of the code meet their specifications and perform as expected.
However, since most of the code is not an independent unit, unit tests have a smaller role in testing the application.
This is because KubeLB is a Kubernetes operator and thus relies on interaction with the Kubernetes API for the most part.
\\
Go supports unit tests and provides the testing package for this purpose.
Each test is passed a testing object which gets informed about the result and unifies the execution and logging.
\\
\autoref{lst:desired-endpoint-test} shows a part of the unit test for the \textit{EndpointIsDesiredState()} function of the \textit{Endpoints} struct, which is used in \autoref{lst:nodecontroller-reconcile}, to check for node changes.
Since the struct and its functions are not part of Kubernetes or the controller runtime, they can be tested individually.
The test consists of several test scenarios that are executed, and the result of which is checked for the expected value.
\textit{createNodeList()} is a helper function that creates a NodeList with one node for each specified address and sets the address to the passed type.
\\
What exactly is tested can be read from the name of the test cases.

\begin{lstlisting}[caption={Unit test for \textit{EndpointIsDesiredState()} function}, label={lst:desired-endpoint-test}]
package kubelb

import (
	corev1 "k8s.io/api/core/v1"
	v1 "k8s.io/apimachinery/pkg/apis/meta/v1"
	"testing"
)

func TestEndpoints_EndpointIsDesiredState(t *testing.T) {
	type fields struct {
		ClusterEndpoints    []string
		EndpointAddressType corev1.NodeAddressType
	}
	type args struct {
		desired *corev1.NodeList
	}
	tests := []struct {
		name   string
		fields fields
		args   args
		want   bool
	}{
		{
			name: "scenario 1: endpoint is desired state",
			fields: fields{
				ClusterEndpoints:    []string{
					"127.0.0.1",
				},
				EndpointAddressType: corev1.NodeExternalIP,
			},
			args: args{
				desired: createNodeList([]string{"127.0.0.1"}, corev1.NodeExternalIP) ,

			},
			want: true,
		},
		{
			name: "scenario 2: endpoint is not in desired state",
			fields: fields{
				ClusterEndpoints:    []string{
					"127.0.0.1",
				},
				EndpointAddressType: corev1.NodeInternalIP,
			},
			args: args{
				desired: createNodeList([]string{"127.0.0.2"}, corev1.NodeInternalIP) ,

			},
			want: false,
		},
		{
			name: "scenario 3: endpoint is in desired state for multiple Node Addresses",
			fields: fields{
				ClusterEndpoints:    []string{
					"127.0.0.1",
					"127.0.0.2",
				},
				EndpointAddressType: corev1.NodeExternalIP,
			},
			args: args{
				desired: createNodeList([]string{"127.0.0.1", "127.0.0.2"}, corev1.NodeExternalIP) ,
			},
			want: true,
		},
		{
			name: "scenario 4: endpoint is not in desired state for multiple Node Addresses",
			fields: fields{
				ClusterEndpoints:    []string{
					"127.0.0.1",
					"127.0.0.2",
				},
				EndpointAddressType: corev1.NodeExternalIP,
			},
			args: args{
				desired: createNodeList([]string{"127.0.0.1", "127.0.0.2", "127.0.0.3"}, corev1.NodeExternalIP) ,
			},
			want: false,
		},
		{
			name: "scenario 5: endpoint is not in desired state as the internal ip is used",
			fields: fields{
				ClusterEndpoints:    []string{
					"127.0.0.1",
				},
				EndpointAddressType: corev1.NodeInternalIP,
			},
			args: args{
				desired: &corev1.NodeList{
					TypeMeta: v1.TypeMeta{},
					ListMeta: v1.ListMeta{},
					Items:    []corev1.Node{{
						TypeMeta:   v1.TypeMeta{},
						ObjectMeta: v1.ObjectMeta{},
						Spec:       corev1.NodeSpec{},
						Status:     corev1.NodeStatus{
							Addresses: []corev1.NodeAddress{
								{
									Type:    corev1.NodeExternalIP,
									Address: "127.0.0.1",
								},
								{
									Type:    corev1.NodeInternalIP,
									Address: "127.0.0.2",
								},
							},
						},
					},
					},
				} ,
			},
			want: false,
		},
	}
	for _, tt := range tests {
		t.Run(tt.name, func(t *testing.T) {
			r := &EndpointsEndpoints{
				ClusterEndpoints:    tt.fields.ClusterEndpoints,
				EndpointAddressType: tt.fields.EndpointAddressType,
			}
			if got := r.EndpointIsDesiredState(tt.args.desired); got != tt.want {
				t.Errorf("EndpointIsDesiredState() = %v, want %v", got, tt.want)
			}
		})
	}
}
\end{lstlisting}


\section{Controller test}\label{sec:controller-test}

As explained in \autoref{sec:unit-test}, unit tests cannot cover the code of the controllers.
To address this issue, the controller-runtime module comes with the envtest\footnote{https://pkg.go.dev/sigs.k8s.io/controller-runtime/pkg/envtest} package that enables testing the controllers, by creating a fake Kubernetes API.
As test framework KubeLB uses Ginko\footnote{https://onsi.github.io/ginkgo/} and Gomega\footnote{https://onsi.github.io/gomega/} for the controller tests, which is recommended by the Kubebuilder authors.
%Todo: check if cite above is needed for recommendation note
In short Ginko allows the developer to write tests of BDD \footnote{https://wikipedia.org/wiki/behavior-driven\_development} style and Gomega is used as a Matcher.
\\
When the controller test suite is started the \textit{BeforeSuite()} function runs once at the beginning.
It initializes the fake Kubernetes api through the envtest package, by setting the CRD and providing a client configuration.
The client configuration is than used for the Manager which therefore creates the actual client for the controller.
\begin{lstlisting}[caption={TCPLoadBalancer controller integration test}, label={lst:controller-test-setup}]

var _ = BeforeSuite(func() {
    logf.SetLogger(zap.New(zap.UseDevMode(false)))

    By("bootstrapping test environment")
    testEnv = &envtest.Environment{
        CRDDirectoryPaths: []string{filepath.Join("..", "..", "..", "config", "crd", "bases")},
    }

    var err error
    cfg, err = testEnv.Start()
    Expect(err).ToNot(HaveOccurred())
    Expect(cfg).ToNot(BeNil())

    err = kubelbk8ciov1alpha1.AddToScheme(scheme.Scheme)
    Expect(err).NotTo(HaveOccurred())

    ctrl.SetLogger(klogr.New())
    ctx := ctrl.SetupSignalHandler()

    envoyServer, err = envoy.NewServer(":8001", true)

    Expect(err).ToNot(HaveOccurred())

    //+kubebuilder:scaffold:scheme

    k8sManager, err := ctrl.NewManager(cfg, ctrl.Options{
        Scheme: scheme.Scheme,
    })
    Expect(err).ToNot(HaveOccurred())

    err = (&TCPLoadBalancerReconciler{
        Client:         k8sManager.GetClient(),
        Cache:          k8sManager.GetCache(),
        Scheme:         k8sManager.GetScheme(),
        EnvoyCache:     envoyServer.Cache,
        EnvoyBootstrap: envoyServer.GenerateBootstrap(),
    }).SetupWithManager(k8sManager, ctx)
    Expect(err).ToNot(HaveOccurred())

    k8sClient = k8sManager.GetClient()
    Expect(k8sClient).ToNot(BeNil())

    go func() {
        err = k8sManager.Start(ctx)
        Expect(err).ToNot(HaveOccurred())
    }()

}, 60)
\end{lstlisting}

With this set up the controller is using the fake api, which is also used inside the test to trigger the reconcile event and check for the expected outcome.
In \autoref{lst:controller-test} a new TCPLoadBalancer is created, and the test checks if the corresponding deployment and service are created by the controller.
The deployment consist of an envoy that also needs the correct configuration what is checked as well.
%Todo: describe limitations of fake api

\begin{lstlisting}[caption={TCPLoadBalancer controller integration test}, label={lst:controller-test}]
var _ = Describe("TcpLb deployment and service creation", func() {

	// Define utility constants for object names and testing timeouts/durations and intervals.
	const (
		tcpLbName      = "serious-application"
		tcpLbNamespace = "default"

		timeout  = time.Second * 10
		interval = time.Millisecond * 250
	)

	var lookupKey = types.NamespacedName{Name: tcpLbName, Namespace: tcpLbNamespace}
	var ctx = context.Background()

	Context("When creating a TcpLoadBalancer", func() {
		var tcpLb = GetDefaultTcpLoadBalancer(tcpLbName, tcpLbNamespace)
		It("Should create an envoy deployment", func() {

			Expect(k8sClient.Create(ctx, tcpLb)).Should(Succeed())

			By("creating a new deployment")

			createdDeployment := &appsv1.Deployment{}

			Eventually(func() bool {
				err := k8sClient.Get(ctx, lookupKey, createdDeployment)
				return err == nil
			}, timeout, interval).Should(BeTrue())

			Expect(createdDeployment.Spec.Template.Spec.Containers[0].Args[1]).Should(Equal(envoyServer.GenerateBootstrap()))
			Expect(createdDeployment.OwnerReferences[0].Name).Should(Equal(tcpLbName))

			By("creating a corresponding service")

			createdService := &corev1.Service{}

			Eventually(func() bool {
				err := k8sClient.Get(ctx, lookupKey, createdService)
				return err == nil
			}, timeout, interval).Should(BeTrue())

			Expect(createdDeployment.Spec.Template.Labels[kubelb.LabelAppKubernetesName]).Should(Equal(createdService.Spec.Selector[kubelb.LabelAppKubernetesName]))
			Expect(createdService.OwnerReferences[0].Name).Should(Equal(tcpLbName))

			By("creating an envoy snapshot")

			snapshot, err := envoyServer.Cache.GetSnapshot(tcpLbName)
			Expect(err).ToNot(HaveOccurred())

			Expect(reflect.DeepEqual(snapshot, envoycp.MapSnapshot(tcpLb, "0.0.1"))).To(BeTrue())

		})
	})
})
\end{lstlisting}
